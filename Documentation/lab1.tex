\documentclass[10pt,onecolumn]{article}
\usepackage[utf8]{inputenc}
 
\pagenumbering{arabic}
%\usepackage{KJN}
\usepackage{siunitx}
\usepackage{graphicx}
\usepackage{placeins}
\usepackage{adjustbox}
\usepackage{tablefootnote}
\usepackage{mathtools}
\usepackage[margin=0.75in]{geometry} %This is for all the margins

\def\@maketitle{%
  \null
  \vskip 2em%
  \begin{center}%
  \let \footnote \thanks
    {\LARGE \@title \par}%
    \vskip 1.5em%
    {\large
      \lineskip .5em%
      \begin{tabular}[t]{c}% <------
        \@author%            <------ Authors
      \end{tabular}\par}%    <------
    \vskip 1em%
  %  {\large \@date}%
  \end{center}%
  \par
  \vskip 1.5em}

\date{23/02/2019}

\title{\vspace{-2.2cm} \textbf{ELEN4020: Data Intensive Computing \\ Laboratory Exercise 1}}
\author{\begin{tabular}{ll}
  Lynch Mwaniki & 1043475 \\
  "" & 1111111 \\
  "" & 1111111 \\
\end{tabular}
 }


\begin{document}
%\centering
\title{\Large{\textbf{ ELEN4020: Data Intensive Computing \\ }}}


\maketitle
\maketitle
\thispagestyle{empty}\pagestyle{empty}
\vspace{-8mm}

\section*{3D Matrix Multiplication}
A commonly used approach for accessing elements in a multidimensional array involves iterating with a number of nested loops. The number of loops required is equal to the dimensions of the array. This approach is computationally expensive but allows accessing the elements of a static array. Adapting this method for dynamic arrays raises the computation time and space complexity and is difficult to implement.\\

\noindent mnb
\noindent mmn

\subsection*{rank3TensorMult}

\subsection*{sd}

\end{document}